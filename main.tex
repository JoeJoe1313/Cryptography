\documentclass{article}
\usepackage[utf8]{inputenc}
\usepackage[bulgarian]{babel}
\usepackage[document]{ragged2e}
\usepackage[dvipsnames]{xcolor}
\usepackage{amsmath}

\title{ЗАДАЧИ ПО КРИПТОГРАФИЯ}
\author{Йоана Левчева\\ Приложна математика, 4 курс, ф.н. 31492}

\begin{document}

\maketitle

\section*{Задача 1}
\justify

Ще започнем, разбивайки по двойки букви открития текст и съответстващия му криптотекст:

\begin{center}
\begin{tabular}{ c c c c c c c c c c c c }

 TH & EW & IN & TE & RO & FO & UR & DI & SC & ON & TE & NT \\ 
 WG & NZ & DZ & WN & IS & OS & BH & GR & RE & AZ & WN & TW    

\end{tabular}
\end{center}

\justify
Сега, ако разгледаме двойката NT $\rightarrow$ TW, става ясно, че NTW се намират на един ред или на един стълб. От TE $\rightarrow$ WN, тъй като NTW се намират на един ред или на един стълб, това означава, че ENTW се намират на един ред или на един стълб. От EW $\rightarrow$ NZ, тъй като ENW се намират на един ред или на един стълб, излиза, че ENTWZ се намират на един ред или на сълб. Нека приемем, че се намират на един ред и ги поставим в първия ред на таблицата ни 5x5, представляваща ключа, за която сме приели, че J ще съвпада с I. В реда ENTWZ се изпълняват условията на трите, разгледани досега двойки букви. Ключът придобива следния вид:

\begin{center}
\begin{tabular}{|c|c|c|c|c|}
        \hline
        E & N & T & W & Z \\
        \hline
         &  &  &  &  \\
        \hline
         &  &  &  &  \\
        \hline
         &  &  &  &  \\
        \hline
         &  &  &  &  \\
        \hline
        \end{tabular}
\end{center}

\justify
От FO $\rightarrow$ OS следва, че FOS също са на един ред или на един стълб, а от ON $\rightarrow$ AZ следва, че те образуват правоъгълник AOZN, като А се намира под N и O се намира под Z. Тъй като Z се намира в края на реда, О е под него и FOS са на един ред (възможността да са на един стълб отпада заради правоъгълника AOZN), то излиза, че S трябва да се намира под E и F се намира под W. Нека попълним втория ред на таблицара, взимайки предвид тези заключения. Получаваме:

\begin{center}
\begin{tabular}{|c|c|c|c|c|}
        \hline
        E & N & T & W & Z \\
        \hline
        S & A &  & F & O \\
        \hline
         &  &  &  &  \\
        \hline
         &  &  &  &  \\
        \hline
         &  &  &  &  \\
        \hline
        \end{tabular}
\end{center}

\justify
Сега от TH $\rightarrow$ WG се образува правоъгълник TWHG и следователно H се намира под W и G се намира под T. Нека ги поставим на третия ред от таблицата. IN $\rightarrow$ DZ също образува правоъгълник IDNZ и следователно I се намира под Z и D се намира под N. От DI $\rightarrow$ GR следва, че DGIR се намират на един ред, откъдето следва, че на третия ред се намират RDGHI. За R остава да се намира под S. Ключът вече има следния вид: 

\begin{center}
\begin{tabular}{|c|c|c|c|c|}
        \hline
        E & N & T & W & Z \\
        \hline
        S & A &  & F & O \\
        \hline
        R & D & G & H & I \\
        \hline
         &  &  &  &  \\
        \hline
         &  &  &  &  \\
        \hline
        \end{tabular}
\end{center}

\justify
От SC $\rightarrow$ RE, тъй като ESR следва, че и C се намира в същия първи стълб и понеже C $\rightarrow$ Е то C се намира на последния ред. Остана да разгледаме само UR $\rightarrow$ BH, при което се образува правоъгълника UBRH, тоест U се намира под H и B се намира под R. Понеже за B има единствена възможност да е на предпоследния ред, то и U излиза, че трябва да се намира на предпоследния ред. Използвайки открититя текст и съответстващия му криптотекст, успяхме да конструираме ключа до следния вид: 

\begin{center}
\begin{tabular}{|c|c|c|c|c|}
        \hline
        E & N & T & W & Z \\
        \hline
        S & A &  & F & O \\
        \hline
        R & D & G & H & I \\
        \hline
        B &  &  & U &  \\
        \hline
        C &  &  &  &  \\
        \hline
        \end{tabular}
\end{center}

\justify
Нека сега разгледаме криптотекста, който трябва да дешифрираме и дешифрираме каквото можем, използвайки отчасти конструрирания ключ: 

\begin{center}
\begin{tabular}{|c|c|c|c|c|c|c|c|c|c|c|c|c|c|}
        \hline
        EB & QX & ZL & HD & LK & IV & QG & OM & AL & EB & VB & DO & SG & SF \\
        \hline
        CR &  &  & GR &  &  &  &  &  & CR &  & IA &  & 0- \\
        \hline
        \end{tabular}
\end{center}

\begin{center}
\begin{tabular}{|c|c|c|c|c|c|c|c|c|c|c|c|c|c|}
        \hline
        ZR & AN & DA & MO & LB & SE & EL & SO & ZL & KD & CO & ZF & GS & IN \\
        \hline
        EI & N- & AN &  &  & EC &  & OF &  &  & -S & WO & R- & DZ \\
        \hline
        \end{tabular}
\end{center}

\justify
Може да предположим, че първата дума от криптотекста е CRYPTOGRAPHY.

\begin{center}
\begin{tabular}{|c|c|c|c|c|c|c|c|c|c|c|c|c|c|}
        \hline
        EB & QX & ZL & HD & LK & IV & QG & OM & AL & EB & VB & DO & SG & SF \\
        \hline
        CR & \textcolor{red}{YP} & \textcolor{red}{TO} & GR & \textcolor{red}{AP} & \textcolor{red}{HY} &  &  &  & CR &  & IA &  & 0- \\
        \hline
        \end{tabular}
\end{center}

\justify
Забелязваме, че ZL $\rightarrow$ TO, тоест се образува правоъгълник ZTLO и мястото на L става известно. А от IV $\rightarrow$ HY от правоъгълника IHVY имаме, че V e под H и Y е под I. Понеже U е точно под H следва, че V и Y се намират на последния ред. Получаваме за ключа:

\begin{center}
\begin{tabular}{|c|c|c|c|c|}
        \hline
        E & N & T & W & Z \\
        \hline
        S & A & \textcolor{red}{L} & F & O \\
        \hline
        R & D & G & H & I \\
        \hline
        B &  &  & U &  \\
        \hline
        C &  &  & \textcolor{red}{V} & \textcolor{red}{Y} \\
        \hline
        \end{tabular}
\end{center}

\justify
Използвайки последния вариант на ключа, се опитваме да дешифрираме още от криптотекста.

\begin{center}
\begin{tabular}{|c|c|c|c|c|c|c|c|c|c|c|c|c|c|}
        \hline
        EB & QX & ZL & HD & LK & IV & QG & OM & AL & EB & VB & DO & SG & SF \\
        \hline
        CR & \textcolor{red}{YP} & \textcolor{red}{TO} & GR & \textcolor{red}{AP} & \textcolor{red}{HY} &  &  & \textcolor{blue}{SA} & CR & \textcolor{blue}{UC} & IA & \textcolor{blue}{LR} & 0\textcolor{blue}{L} \\
        \hline
        \end{tabular}
\end{center}

\begin{center}
\begin{tabular}{|c|c|c|c|c|c|c|c|c|c|c|c|c|c|}
        \hline
        ZR & AN & DA & MO & LB & SE & EL & SO & ZL & KD & CO & ZF & GS & IN \\
        \hline
        EI & N- & AN &  & \textcolor{blue}{S}- & EC & \textcolor{blue}{TS} & OF & \textcolor{blue}{TO} &  & \textcolor{blue}{Y}S & WO & R\textcolor{blue}{L} & DZ \\
        \hline
        \end{tabular}
\end{center}

\justify
От QX $\rightarrow$ YP имаме, че Q е на последния ред, а P и X на предпоследния, по-конкретно X e над Y. От LK $\rightarrow$ AP щом P е на предпоследния ред следва, че и К е на предпоследния ред, като K е под D и P e под X. Следователно Q е на последния ред под P. И остава М да е под K. Вече разпоалагме с целия ключ: 

\begin{center}
\begin{tabular}{|c|c|c|c|c|}
        \hline
        E & N & T & W & Z \\
        \hline
        S & A & \textcolor{red}{L} & F & O \\
        \hline
        R & D & G & H & I \\
        \hline
        B & \textcolor{blue}{K} & \textcolor{blue}{P} & U & \textcolor{blue}{X} \\
        \hline
        C & \textcolor{blue}{M} & \textcolor{blue}{Q} & \textcolor{red}{V} & \textcolor{red}{Y} \\
        \hline
        \end{tabular}
\end{center}

\justify
Разполагайки с ключа, можем да дешифрираме целия криптотекст:

\begin{center}
\begin{tabular}{|c|c|c|c|c|c|c|c|c|c|c|c|c|c|}
        \hline
        EB & QX & ZL & HD & LK & IV & QG & OM & AL & EB & VB & DO & SG & SF \\
        \hline
        CR & \textcolor{red}{YP} & \textcolor{red}{TO} & GR & \textcolor{red}{AP} & \textcolor{red}{HY} & \textcolor{olive}{PL} & \textcolor{olive}{AY} & \textcolor{blue}{SA} & CR & \textcolor{blue}{UC} & IA & \textcolor{blue}{LR} & 0\textcolor{blue}{L} \\
        \hline
        \end{tabular}
\end{center}

\begin{center}
\begin{tabular}{|c|c|c|c|c|c|c|c|c|c|c|c|c|c|}
        \hline
        ZR & AN & DA & MO & LB & SE & EL & SO & ZL & KD & CO & ZF & GS & IN \\
        \hline
        EI & N\textcolor{olive}{M} & AN & \textcolor{olive}{YA} & \textcolor{blue}{S}\textcolor{olive}{P} & EC & \textcolor{blue}{TS} & OF & \textcolor{blue}{TO} & \textcolor{olive}{DA} & \textcolor{blue}{Y}S & WO & R\textcolor{blue}{L} & DZ \\
        \hline
        \end{tabular}
\end{center}

\justify
Z в края на изрчението играе роля на доплъваща буква до четен брой букви на изречението без да променя смисъла на думата. Окончателно криптотекстът се дешифрира като: \\

\justify
CRYPTOGRAPHY PLAYS A CRUCIAL ROLE IN MANY ASPECTS OF TODAY'S WORLD

\section*{Задача 2}

\justify
Първо ще отбележим, че $\mathbf{Z_{26}}$ $\simeq$ $\mathbf{Z_2}$ $\times$ $\mathbf{Z_{13}}$. Нека разгледаме матрицата $K$ = $\begin{pmatrix}
                             a & b \\
                             c & d
                         \end{pmatrix}$
                         $\in$ $\mathbf{Z_{13}}$. 
По условие $K = K^{-1}$. Следователно $KK^{-1} = KK = K^2 = I$. Това е еквивалентно на следната система: 

\begin{center}
\begin{tabular}{ c }

 $a^2 + bc = 1$ \\ 
 $b(a + d) = 0$ \\
 $c(a + d) = 0$ \\
 $d^2 + bc = 1$
 
\end{tabular}
\end{center}

\justify
Ако $a + d \neq 0$ следва, че $b = 0$ и $c = 0$ и също $a^2 = 1$ и $d^2 = 1$. От тук получаваме две решения. \\
\\Сега, ако $a + d = 0$, имаме следните случаи:

$\bullet$ $a = 0$. Тогава $bc = 1$, което означава, че $b$ и $c$ са обратими и следователно имаме $13 - 1 = 12$ още $12$ решения, защото в $\mathbf{Z_{13}}$ има 12 обратими елемента.

$\bullet$ $a = 1$. Тогава $bc = 0$, т.е. имаме още 2*13 - 1 = 25 решения.

$\bullet$ $a = -1$. Аналогично, $bc = 0$ и имаме още 2*13 -1 = 25 решения.

$\bullet$ $a \neq 0, 1, -1$. Тогава $c$ = $(1 - a^2)b^{-1}$. Това ни дава още $(13-1)(13-3) = 120$ решения.

\justify
Сумирайки всички решения, получаваме, че в $\mathbf{Z_{13}}$ има $2 + 12 + 25 + 25 + 120 = 184$ решения.

\justify
Остава да преброим матриците с това свойство над $\mathbf{Z_2}$. Аналогично, нека разгледаме матрицата $K$ = $\begin{pmatrix}
                             a & b \\
                             c & d
                         \end{pmatrix}$
                         $\in$ $\mathbf{Z_{2}}$. 
Отново получаваме същата система като при $\mathbf{Z_{13}}$

\justify
Ако $a + d \neq 0$ следва, че $b = 0$ и $c = 0$ и също, че или $a = 1$ или $d = 1$. Но от $b = 0$ и $c = 0$ следва, че и $a^2 = 1$ и $d^2 = 1$, т.е. $a = 1$ и $d = 1$, което е невъзможно. Следователно от тук получаваме 0 решения. \\
\\Сега, ако $a + d = 0$, имаме следните случаи: 

$\bullet$ $a = 0$. Тогава следва, че $d = 0$ и $bc = 1$, което означава, че $b$ и $c$ са обратими и следователно имаме една взъможност $b = 1$ и $c = 1$. Този случай ни дава 1 решение.

$\bullet$ $a = 1$. Тогава и $d = 1$ и още $bc = 0$, т.е. или $b = 0$ или $c = 0$ или $b = c = 0$. От тук имаме още 3 решения.

\justify
Сумирайки всички решения, получаваме, че в $\mathbf{Z_{2}}$ има $1 + 3 = 4$ решения.

\justify
Тъй като $\mathbf{Z_{26}}$ $\simeq$ $\mathbf{Z_2}$ $\times$ $\mathbf{Z_{13}}$, то над $\mathbf{Z_{26}}$ има 4*184 = 736 матрици със свойство $K = K^{-1}$.

\section*{Задача 3}

\justify
Ще започнем с това да запишем кое число съответства на всяка буква от открития текст и съответстващия му криптотекст, използвайки дадената схема за кодиране:

\begin{center}
\begin{tabular}{|c|c|c|c|c|c|c|c|c|c|c|c|}
        \hline
        C & R & Y & P & T & O & G & R & A & P & H & Y \\
        \hline
        2 & 17 & 24 & 15 & 19 & 14 & 6 & 17 & 0 & 15 & 7 & 24 \\
        \hline
        \end{tabular}
\end{center}

\begin{center}
\begin{tabular}{|c|c|c|c|c|c|c|c|c|c|c|c|}
        \hline
        V & G & Y & X & A & R & D & I & G & L & M & L \\
        \hline
        21 & 6 & 24 & 23 & 0 & 17 & 3 & 8 & 6 & 11 & 12 & 11 \\
        \hline
        \end{tabular}
\end{center}

\end{document}

